%% Dokumenteinstellungen %%%%%%%%%%%%%%%%%%%%%%%%%%%%%%%%%%%%
\documentclass[a4paper,oneside,12pt]{scrartcl}

%% Deutsche Anpassungen %%%%%%%%%%%%%%%%%%%%%%%%%%%%%%%%%%%%%
%\usepackage[ngerman]{babel}
\usepackage[T1]{fontenc}
\usepackage[ansinew]{inputenc}
\usepackage{lmodern} %Type1-Schriftart f�r nicht-englische Texte

%% Packages f�r Grafiken & Abbildungen %%%%%%%%%%%%%%%%%%%%%%
\usepackage{graphicx} %%Zum Laden von Grafiken
%\usepackage{subfig} %%Teilabbildungen in einer Abbildung
%\usepackage{tikz} %%Vektorgrafiken aus LaTeX heraus erstellen


%% Packages f�r Formeln %%%%%%%%%%%%%%%%%%%%%%%%%%%%%%%%%%%%%
\usepackage{amsmath}
\usepackage{amsthm}
\usepackage{amsfonts}


%% Andere Packages %%%%%%%%%%%%%%%%%%%%%%%%%%%%%%%%%%%%%%%%%%
%\usepackage{a4wide} %%Kleinere Seitenr�nder = mehr Text pro Zeile.
\usepackage{fancyhdr} %%Fancy Kopf- und Fu�zeilen
%\usepackage{longtable} %%F�r Tabellen, die eine Seite �berschreiten
\usepackage{lastpage}
\usepackage[raggedright]{subfigure}

%%%%%%%%%%%%%%%%%%%%%%%%%%%%%%%%%%%%%%%%%%%%%%%%%%%%%%%%%%%%%
%% TODO
%%%%%%%%%%%%%%%%%%%%%%%%%%%%%%%%%%%%%%%%%%%%%%%%%%%%%%%%%%%%%
% 
% 
%%%%%%%%%%%%%%%%%%%%%%%%%%%%%%%%%%%%%%%%%%%%%%%%%%%%%%%%%%%%%



%%%%%%%%%%%%%%%%%%%%%%%%%%%%%%%%%%%%%%%%%%%%%%%%%%%%%%%%%%%%%
%% Optionen / Modifikationen
%%%%%%%%%%%%%%%%%%%%%%%%%%%%%%%%%%%%%%%%%%%%%%%%%%%%%%%%%%%%%
\input{Einstellungen}

%%%%%%%%%%%%%%%%%%%%%%%%%%%%%%%%%%%%%%%%%%%%%%%%%%%%%%%%%%%%%
%% DOKUMENT
%%%%%%%%%%%%%%%%%%%%%%%%%%%%%%%%%%%%%%%%%%%%%%%%%%%%%%%%%%%%%
\begin{document}

\title{Praktikum 4.2: Aktives Filter}
\date{\today}
\author{Cyril Stoller, Marcel}
\maketitle

%% Inhaltsverzeichnis %%%%%%%%%%%%%%%%%%%%%%%%%%%%%%%%%%%%%%%
\tableofcontents %Inhaltsverzeichnis

%\pagestyle{fancy} %%Ab hier die Kopf-/Fusszeilen: headings / fancy / ...

\vspace{2cm}

\begin{abstract}
	
\begin{center}
	Dieser Bericht ist erg�nzend zum Laborjournal und enth�lt vertiefte Diskussion der gemessenen Resultate, sowie eine Beschreibung der
\end{center}
	
\end{abstract}

\vspace{2cm}


%%%%%%%%%%%%%%%%%%%%%%%%%%%%%%%%%%%%%%%%%%%%%%%%%%%%%%%%%%%%%
%%                                                         %%
%%         Kapitel / Hauptteil des Dokumentes              %%
%%                                                         %%
%%%%%%%%%%%%%%%%%%%%%%%%%%%%%%%%%%%%%%%%%%%%%%%%%%%%%%%%%%%%%



\section{Ziel}

Dieser Bericht beinhaltet genaue Angaben zur Durchf�hrung und eine Diskussion des Versuches Aktive Filter im Modul BTE5032.02. Das Ziel ist es, die im letzten Semester erlernte Filtertheorie in der Praxis nachzuvollziehen und zu vertiefen. Dazu wird ein Aktives Filter diemensioniert, aufgebaut und ausgemessen. Die Resultate werden anschliessend mit Resultaten aus einer Simulation verglichen.

\section{Einleitung}
\subsection{Motivation}

\subsection{Aufgabenstellung}
Die Aufgabenstellung ist unter \url{http://moodle.bfh.ch/course/view.php?id=3380} oder im Anhang zu finden.

\section{Durchf�hrung}

\subsection{Theorie}

\subsection{Dimensionierung}
Bei der Dimensionierung sind wir genau nach der Aufgabenstellung gegangen. Zuerst haben wir das \beta berechne.

Um die Werte genau zu erreichen, haben wir bei den Widerst�nden R1 und R3 zus�tzlich ein Potentiometer in serie geschaltet. Somit bleibt einen gewisser Spielraum um die Schaltung sauber abzustimmern.

\subsection{Simulation}
Die Simmulation wurde in LT Spice durchgef�rht. Da in der Aufgabenstellung der OpAmp LM741 vorgegeben ist, mussten wir diesen zuerst noch im Spice hinzuf�gen. 
Danach die Simulation gem�ss dem Schema aufbauen und die Werte entsprechend der Dimensionierung w�hlen. Die Widerst�nde haben wir in der Simulation genau auf den berechneten Wert eingestellt. Bei der realen Schaltung haben wir einen N�herungswert genommen und mit einem Potentiometer in Serie zum Widerstand den Wert genau abgeglichen.


\subsection{Aufbau}
Der Aufbau auf der Steckplatte hat uns am meisten Schwierigkeiten bereitet. Gleich zwei mal hatten wir einen Fehler im Aufbau. 


\subsection{Messung}
Messmittellieste ist im Anhang zu finden.

\subsection{Fehlerabsch�tzung}

\subsection{Diskussion}


\section{Schlussfolgerung}
Als Fazit k�nnen wir sagen, dass wir mit hilfe der Anleitung die Dimensionierung gut durchf�hren konnten. Leider haben wir beim Aufbau zu viel Zeit verloren, die uns danach gefehlt hat, um die Auswertung noch detailierter zu gestalten. So hatten wir zum Beispiel keine Zeit mehr die G�te anhand der Bandbreite und der Mittenfrequenz nachzukontrollieren. 

\section{Literaturverzeichnis}

\section{Anhang}

\end{document}
